%%%%%%%%%%%%%%%%%%%%%%%%%%%%%%%%%%%%%%%%%%%%%%%%%%%%%%%%%%%%%%%%
%%%% Getting help
%%%%%%%%%%%%%%%%%%%%%%%%%%%%%%%%%%%%%%%%%%%%%%%%%%%%%%%%%%%%%%%%
\chapter{Getting help} 

The first step before using any program is to read the manual. All
programs in the \RSAT package come with an on-line help, which is
obtained by typing the name of the program followed by the option
\option{-h}. For example, to get a detailed description of the
functionality and options for the program \texttt{retrieve-seq}, type

\begin{lstlisting}
retrieve-seq -h
\end{lstlisting}

The detailed help is specially convenient before using the program for
the first time. A complementary functionality is offered by the option
\texttt{-help}, which prints a short list of options. Try:

\begin{lstlisting}
retrieve-seq -help
\end{lstlisting}

which is convenient to remind the precise formulation of arguments for
a given program.



