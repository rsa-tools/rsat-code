%%%%%%%%%%%%%%%%%%%%%%%%%%%%%%%%%%%%%%%%%%%%%%%%%%%%%%%%%%%%%%%%
%
% Installation guide for regulatory Sequence Analysis Tools
%
%%%%%%%%%%%%%%%%%%%%%%%%%%%%%%%%%%%%%%%%%%%%%%%%%%%%%%%%%%%%%%%%

%\documentclass{book}
\documentclass[12pt,a4paper, oneside]{scrreprt} % KOMA-class Neukam and Kohm, scrbook alternatively
%\documentstyle[makeidx]{book}
\makeindex
\usepackage{color}
\usepackage{times}
\usepackage{graphics}
\usepackage{latexsym}
\usepackage{makeidx}


%%%%%%%%%%%%%%%%%%%%%%%%%%%%%%%%%%%%%%%%%%%%%%%%%%%%%%%%%%%%%%%%
%%%%%%%%%%%%%%%%%%%%%%%%%%% commands %%%%%%%%%%%%%%%%%%%%%%%%%%%
\newcommand{\RSAT}{\textbf{\textit{RSAT}}}
\newcommand{\file}[1]{\textit{#1}}
\newcommand{\concept}[1]{\index{#1}\textsl{#1}}
\newcommand{\command}[1]{\begin{footnotesize}\begin{quote}\textcolor{blue}{\texttt{#1}}\end{quote}\end{footnotesize}}
\newcommand{\result}[1]{\begin{footnotesize}\begin{quote}\textcolor{green}{\texttt{#1}}\end{quote}\end{footnotesize}}
\newcommand{\program}[1]{\textbf{\textsl{#1}}}
\newcommand{\option}[1]{\texttt{#1}}
\newcommand{\email}[1]{\textit{#1}}

\newcommand{\address}[1]{\small{#1}}
\newcommand{\org}[1]{\textit{#1}}
\newcommand{\gene}[1]{\textit{#1}}

\newcommand{\url}[1]{\textit{#1}}
\newcommand{\urlref}[1]{\footnote{\textit{#1}}}

\newcommand{\scmbb}{
	Service de Conformation des Macromol\'{e}cules Biologiques et de Bioinformatique, \\
	Universit\'{e} Libre de Bruxelles, \\
	Campus Plaine, CP 263, Boulevard du Triomphe, B-1050 Bruxelles, Belgium. \\
	Tel: +32 2 650 2013 - Fax: +32 2 650 5425
}

%%%%%%%%%%%%%%%%%%%%%%%%%%%%%%%%%%%%%%%%%%%%%%%%%%%%%%%%%%%%%%%%
%%%%%%%%%%%%%%%%%%%%%%%%% environments %%%%%%%%%%%%%%%%%%%%%%%%%




\begin{document}

\RSATtitlePage{Installation guide}

% \title{Regulatory Sequence Analysis Tools \\
% Installation guide}
% \author{
% 	Jacques van Helden \\
% 	\email{jvhelden@ulb.ac.be} \\
% 	\bigre
% }
% \maketitle

\newpage
\tableofcontents
\newpage

\chapter{Description and requirements}

\section{Description}

This documents describes the installation procedure for the software
suite \textbf{Regulatory Sequence Analysis Tools} (\RSAT), which
offers several dozens of tools to detect cis-regulatory elements in
DNA sequences
\cite{Thomas-Chollier:2008:W119-27,vanHelden:2003:3593-6,vanHelden:2000:177-87}.

\section{Requirements}

\subsection{Operating system}

\RSAT is a unix-based package. It has been installed successfully on
the following operating systems.

\begin{enumerate}
\item Linux

\item Mac OSX

\item Sun Solaris

\item Dec Alpha

\end{enumerate}

\RSAT is not compatible with any version of Microsoft Windows and I
have no intention to make it compatible in a foreseeable future. Since
most programs are written in perl, part of them might run under
windows, but some others will certainly not, because they include
calls to unix system commands.

\subsection{Perl language}

Most of the programs in \RSAT are written in perl. Version 5.1 or
later is recommended. A set of Perl modules is required, the \RSAT
package includes a script to install them automatically (see
Chapter~\ref{chap:perl_modules}).

\subsection{Python language}

Some of the programs in \RSAT are written in python. Version 2.4 or
later is recommended.

\subsection{Java language}

Some of the NeAT tools are written in Java. Java Runtime Environment 5
or higher is recommended.

\subsection{Helper applications}

\subsubsection{wget}

The program \program{wget}, is used  to download
\begin{enumerate}

\item some helper programs developed by third-parties, which can be
  installed in \RSAT;

\item genomes from the \RSAT server to your local \RSAT installation.

\end{enumerate}

\program{wget} is part of linux distribution. If it is not installed
on your computer, you can download it from
\url{http://www.gnu.org/software/wget/}.  An installation package for
Mac OSX can be found at
\url{http://download.cnet.com/Wget/3000-18506_4-128268.html}.


\subsubsection{gnuplot}

The standard version of the \RSAT program \program{XYgraph} export
figures in bitmapformat (png, jpeg). If you want to support vectorial
drawings (pdf), which give a much better resolution for printing, you
need to install the freeware software \program{gnuplot} (4.2 or
later), which can be downloaded from \url{http://www.gnuplot.info/}.



%%%%%%%%%%%%%%%%%%%%%%%%%%%%%%%%%%%%%%%%%%%%%%%%%%%%%%%%%%%%%%%%
% Installation
\chapter{Obtaining \RSAT distribution}

For the time being, \RSAT is distributed as a compressed archive. In a
near future, we will also distribute it via an anonymous CVS server,
which will greatly facilitate the updates.

\textbf{Note} The CVS distribution will soon be available for external
users, but we still need to configure the CVS server to accept a guest
login. For the time being, the CVS distribution is still restricted to
the people from the lab. Inbetween, the only distribution mode for
external users is the compressed archive. If you are not member of the
BiGRe laboratory, please skip the section \textit{Installation from
  the CVS repository}.

\section{Downloading \RSAT from the CVS server}

\textbf{Warning:} we currently cannot manage user profiles in our CVS
server, so the \RSAT code is still distributed as a compressed
archive. Please skip this section if you downloaded a compessed
archive from the Web server (a file named
\file{rsa-tools\_2009XXXX.tar.gz}, where XXXX indicates the date).

\subsection{CVS configuration}

You need to indicate your \program{cvs} client to use \program{ssh}
as remote shell application. For this, you can specify an environment
variable.

\emph{If your shell is tcsh}, add the following line to the
\file{.cshrc} file in your home directory.

\begin{lstlisting}
setenv CVS_RSH ssh
\end{lstlisting}


\emph{If your shell is bash}, add the following line to the
\file{.bashrc} file in your home directory.

\begin{lstlisting}
export CVS_RSH=ssh
\end{lstlisting}



\subsection{Obtaining a first version of \RSAT programs}

The following command should be used the first time you retrieve the
tools from the server (you need to replace \texttt{[mylogin]} by the
login name you received when signing the \RSAT license).

\begin{lstlisting}
cvs -d [mylogin]@cvs.bigre.ulb.ac.be:/cvs/rsat co rsa-tools
\end{lstlisting}


This will create a directory \file{rsa-tools} on your computer, and
store the programs in it. Note that at this stage the programs are not
yet functional, because you still need to install genomes, which are
not included in the CVS distribution.

\subsection{Updating \RSAT programs}

Once the tools have been retrieved, you can obtain updates very
easily. For this, you need to change your directory to the rsa-tools
directory, and use the \texttt{cvs} command in the following way.

\begin{lstlisting}
cd rsa-tools
cvs update -d
\end{lstlisting}

\section{Installation from a compressed archive}

Uncompress the archive containing the programs. The archive is
distributed \texttt{tar} format.

The \texttt{.tar.gz} file can be uncompressed with the command
\program{tar}, which are part of the default unix installation.

\begin{lstlisting}
tar -xpzf rsa-tools_yyyymmdd.tar.gz
\end{lstlisting}

\chapter{Initializing \RSAT}

\section{Configuring \RSAT}


In order to use the command-line version of \RSAT, you first need an
account on a Unix machine where \RSAT has been installed, and you
should know the directory where the tools have been installed (if you
don't know, ask assistance to your system administrator).

In the following instruction, we will assume that \RSAT is installed
in the directory \texttt{/home/rsat/rsa-tools}. This path has to be
replaced by the actual path where \RSAT has been installed on your
computer.

%%%%%%%%%%%%%%%%%%%%%%%%%%%%%%%%%%%%%%%%%%%%%%%%%%%%%%%%%%%%%%%%
%%%% Accessing the programs
%%%%%%%%%%%%%%%%%%%%%%%%%%%%%%%%%%%%%%%%%%%%%%%%%%%%%%%%%%%%%%%%
\subsection{Adding \RSAT  to your path}

Before starting to use the tools, you need to define an environment
variable (\texttt{RSAT}), and to add some directories to your path.

\begin{enumerate}

\item Create an environment variable named \textit{RSAT} and
  containing the path of rsa-tools.

  The way to create an environment variable depends on your shell. To
  know you shell, you can type

\begin{footnotesize}
\begin{verbatim}
echo $SHELL
\end{verbatim}
\end{footnotesize}

The answer should be something like \result{/sbin/bash} or
\result{/bin/tcsh}.

Now, if we assume that \RSAT have been installed in the following
directory.

\begin{footnotesize}
\begin{verbatim}
/home/rsat/rsa-tools
\end{verbatim}
\end{footnotesize}


We will declare the \RSAT path to your shell by defining an
environment variable named \texttt{RSAT}.  We will then add the path
of the \RSAT \ perl scripts, python scripts and binaries to your
path. In addition, add java jar files to your classpath.

\item If your default shell is \textbf{tcsh} or \textbf{csh}, type the
  following commands (you probably need to update the first command to
  specify the RSAT path of your machine.

\begin{footnotesize}
\begin{verbatim}
setenv RSAT /home/rsat/rsa-tools
set path=($path $RSAT/bin)
set path=($path $RSAT/perl-scripts)
set path=($path $RSAT/python-scripts)
set classpath=($classpath $RSAT/java/lib/NeAT_javatools.jar)
rehash
\end{verbatim}
\end{footnotesize}



If your shell is bash, you should type the following command:

\begin{footnotesize}
\begin{verbatim}
export RSAT=/home/rsat/rsa-tools
export PATH=${PATH}:${RSAT}/bin
export PATH=${PATH}:${RSAT}/perl-scripts
export PATH=${PATH}:${RSAT}/python-scripts
export CLASSPATH=${CLASSPATH}:${RSAT}/java/lib/NeAT_javatools.jar
\end{verbatim}
\end{footnotesize}



(the \texttt{rehash} command updates the list of executable programs)

\end{enumerate}

If you are using a different shell than bash, csh or tcsh, the
specification of environment variables might differ from the syntax
above.  In case of doubt, ask your system administrator how to
configure your environment variables and your path.

The specification of the environment variables and paths are required
each time you want to use \RSAT. You can add these specification to
your personal profile.  This file is normally found at the root of
your personal account, in the file \file{.bashrc} if your shell is
bash, or \file{.cshrc} if your shell is csh or tcsh. If you don't know
how to proceed, ask your system administrator.



\subsection{Checking the RSAT path}

 The previous step should have included all the \RSAT programs in
your path.  To check if it worked, just type:

\command{random-seq -l 350}

If your configuration is correct, this command should return a random
sequence of 350 nucleotides.

You are now able to use any program from the \RSAT package, untill you
quit your session. It is however not very convenient to set the path
manually each time you open a new connection. You can modify your
default configuration by including the above commands in the file
\file{.cshrc} (in tcsh) or \file{.bashrc} (in bash) which should be
found at the root of your home directory. If you don't know how to
modify this file, see the system administrator.



%%%%%%%%%%%%%%%%%%%%%%%%%%%%%%%%%%%%%%%%%%%%%%%%%%%%%%%%%%%%%%%%
% Directories

\section{Initializing the directories}

In addition to the programs, the installation of rsa-tools requires
the creation of a few directories for storing data, access logs (for
the web server), and temporary files.

The distribution includes a series of make scripts which will
facilitate this step. You just need go to the rsa-tools directory, and
start the appropriate make file.

\begin{lstlisting}
cd $RSAT
make -f makefiles/init_RSAT.mk init
\end{lstlisting}
% $

%%%%%%%%%%%%%%%%%%%%%%%%%%%%%%%%%%%%%%%%%%%%%%%%%%%%%%%%%%%%%%%%
% Config file
\section{Adapting \RSAT \  local configuration}

The \RSAT distribution comes with a template configuration file named
\file{RSAT\_config\_default.props} and located in the \file{rsa-tools}
directory.

Copy this file to create your own config file \file{RSAT\_config.props}.

\begin{lstlisting}
cp RSAT_config_default.props RSAT_config.props
\end{lstlisting}


\subsection{Editing the \RSAT\ local configuration file}

You need to edit the file \file{RSAT\_config.props} and specify the
parameters of your local configuration. In particular, \emph{it is
  crucial to specify the full path of the variable RSAT}, which
specifies the RSAT main directory.

%%%%%%%%%%%%%%%%%%%%%%%%%%%%%%%%%%%%%%%%%%%%%%%%%%%%%%%%%%%%%%%%
%% Check the RSAT paths
\subsection{Checking the RSAT path}

The \RSAT programs should now be included in your path.  To check if
this is done properly, just type:

\begin{lstlisting}
random-seq -l 350
\end{lstlisting}

If your configuration is correct, this command should return a random
sequence of 350 nucleotides.

Don't worry if you see a warning looking like this:

\begin{footnotesize}
\begin{verbatim}
; WARNING	The tabular file with the list of supported organism cannot be read
; WARNING	Missing file	/no_backup/rsa-tools/public_html/data/supported_organisms.tab
\end{verbatim}
\end{footnotesize}

This warning will disappear after we download the first organism in
\RSAT.

You are now able to use any program from the \RSAT package, untill you
quit your session. It is however not very convenient to set the path
manually each time you open a new connection. You can modify your
default configuration by including the above commands in the file
\file{.cshrc} (in tcsh) or \file{.bashrc} (in bash) which should be
found at the root of your home directory. If you don't know how to
modify this file, see the system administrator.

\subsection{Checking the java path (required for pathway tools only)}

The \program{java} language is necessary for the path finding tools
included in the \program{NeAT} suite. If you want to use those tools,
you can check the correct setting for the java-based pathway tools, by
typing the following command.

\begin{lstlisting}
java graphtools.util.ListTools
\end{lstlisting}

This should display the list of available pathway analysis tools,
which are part of the NeAT package.



\chapter{Installing Perl modules}
\label{chap:perl_modules}

Some Perl modules are required for the graphical tools of \RSAT, and
for some other specific programs. The perl modules can be found in the
Comprehensive Perl Archive Network (\url{http://www.cpan.org/}), or
can be installed with the command \program{cpan}.

\section{Before installing Perl modules: install the GD library}

The Perl module GD.pm requires the prior installation of the GD
library. 

\begin{itemize}

\item On \emph{Linux} systems, this library can be installed with the
  package manager of your distribution (e.g. \command{apt-get}).


\item On \emph{Mac OSX} systems, the installation of the GD library is
  quite tricky. The bet is to follow the installation protocol of the
  libgd Web site: \url{http://www.libgd.org/DOC_INSTALL_OSX}.

\end{itemize}

\section{Automatic installation of Perl modules}

The simplest way to install all the required Perl modules is to tye
the command below. \emph{Beware:} this command sudo requires
administrator rights on the computer. If you don't have the root
password, please consult your system administrator.

\begin{lstlisting}
cd ${RSAT}

## Display the list of Perl modules that will be installed
make -f makefiles/install_RSAT.mk list_perl_modules

## Install the Perl modules
make -f makefiles/install_RSAT.mk install_perl_modules
\end{lstlisting}
%% closing $

\section{Utilization of the Perl modules in \RSAT}

For information, we describe hereafter the list of modules that will
be installed with this command, and the reason why it is useful to
install them before running \RSAT programs.

If you are not interested by technical details, you can skip this
section.


\begin{enumerate}
\item \program{GD.pm} Interface to Gd Graphics Library. Used by
  \program{XYgraph} and \program{feature-map}. 

\item \program{PostScript::Simple} Produce PostScript files from Perl. Used by
  \program{feature-map}.

\item \program{Statistics::Distributions} is used to calculate some probability
  distribution functions. In particular, it is used by the program
  \program{position-analysis} to calculate the P-value of the
  chi2. 

  \textbf{Notes} 
  \begin{itemize}
  \item In previous releases, the chi2 P-value was computed using
    \program{Math::CDF}, but the precision was limited to
    1e-15. \program{Statistics::Distributions} can compute P-values
    down to 1e-65.
  \item For the discrete functions (binomial , Poisson,
    hypergeometric) \RSAT relies on a custom library (
    \$RSAT/perl-scripts/lib/RSAT/stats.pm) which reaches a precision
    of 1e-300.
  \end{itemize}

\item \program{File::Spec}, \program{POSIX} and \program{Data::Dumper}
  are required for some functions of \program{matrix-scan}.

\item \program{Util::Properties} is required to load property files, which are
  used to specify the site-specific configuration of your \RSAT
  server. Property filesa are also useful to write your own perl
  clients for the Web service interface to \RSAT (RSATWS).

\item \program{Class::Std::Fast} and \program{Storable} are required
  to bring persistence to data structures like organisms. This library
  can be easyly installed via CPAN.

\item \program{XML::LibXML} is required for parsing and writing XML and uses
  the XML::Parser::Expat library. It is necessary for some RSAT
  applications.

\item \program{DBD::mysql} and \program{DBI} those two libraries are
  required by the program \program{retrieve-ensembl-seq} in order to
  access the ENSEMBL database.

\item \program{SOAP::WSDL}, \program{Module::Build::Compat} and
  \program{Util::Properties} (already mentioned above) are required
  for the Web services. \RSAT Web services is a convenient interface
  that permits to write Perl scripts to run \RSAT queries on a remote
  server.

\item \program{Bio::Perl} is required for the Ensembl API, which in
  turn is required for handling genomes installed on the ENSEMBL
  database (\url{http://www.ensembl.org/}).

\end{enumerate}


%%%%%%%%%%%%%%%%%%%%%%%%%%%%%%%%%%%%%%%%%%%%%%%%%%%%%%%%%%%%%%%%
%% Compile C/C++ programs
\section{Compiling C programs in \RSAT}

Some of the tools available in \RSAT (\program{info-gibbs},
\program{matrix-scan-quick}, \program{count-words}) are written in the
\program{C} language. The distribution only contains the sources of
these tools, because the binaries are operating system-dependent. The
programs can be compiled in a very easy way.

\begin{lstlisting}
cd ${RSAT}
make -f makefiles/init_RSAT.mk compile_all
\end{lstlisting}

This will compile and install the following programs in the directory
\file{\$RSAT/bin}. 

The installation directory can be changed by redefining the BIN
variable. For instance, if you have the system adminstrator
privileges, you could install the compiled programs in the standard
directory for compiled packages (\file{/usr/local/bin}).

\begin{lstlisting}
cd ${RSAT}
make -f makefiles/init_RSAT.mk compile_all BIN=/usr/local/bin
\end{lstlisting}


\begin{itemize}
\item \program{info-gibbs}: a gibbs sampling algorithm based on
  optimization of the the information content of the motif
  \cite{Defrance:2009}.

\item \program{count-words}: an efficient algorithm for counting word
  occurrences in DNA sequences. This program is much faster than
  \program{oligo-analysis}, but it only returns the occurrences and
  frequencies, whereas oligo-analysis returns over-representation
  statistics and supports many additional
  options. \program{count-words} is routinely used to compute word
  frequencies in large genome sequences, for calibrating the Markov
  models used by \program{oligo-analysis}.

\item \program{matrix-scan-quick}: an efficient algorithm for scanning
  sequences with a position-specific scoring matrix. As its name
  indicates, \program{matrix-scan-quick} is \emph{much} faster than
  the Perl script \program{matrix-scan}, but presents reduced
  functionalities (only computes the weight, returns either a list of
  sites or the weight score distribution).
\end{itemize}


\section{Compiling the Stubb for Web services}

If you want to use the Web services, you need to compile the Stubb in
order to synchronize the Perl modules of your client with the service
specification of the server.


\begin{lstlisting}
## Change directory to the sample Perl Web services clients
cd $RSAT/ws_clients/perl_clients

## Compile the stubb.
## Your computer will open a connection to the Web services server 
## and collect the WSDL specifications of the supported services.
make stubb

## Test the connection to the Web services
make test
\end{lstlisting}

%$

\section{Importing and installing programs from third parties}

In addition to the binaries incorporated in the suite itself, \RSAT
uses some programs developed by third parties for specific
tasks. Those programs can be downloaded and installed automatically
using the makefile \file{install\_rsat.mk}.

Before doing this, you must make sure that the program \program{wget}
(For Linux: \url{http://www.gnu.org/software/wget/}; for Mac OSX
\url{http://download.cnet.com/Wget/3000-18506_4-128268.html}).

\begin{lstlisting}
cd $RSAT
make -f makefiles/install_rsat.mk install_ext_apps
\end{lstlisting}
%$

\section{Libraries required to support EnsEMBL genomes}
\label{sect:ensembl_libraries}

Since 2009, a series of \RSAT programs support a direct access to the
EnsEMBL database in order to ensure a convenient access to genomes
from higher organisms \cite{Sand:2009}.

\begin{itemize}
\item \program{supported-organisms-ensembl}
\item \program{ensembl-org-info}
\item \program{retrieve-ensembl-seq.pl} 
\item \program{get-ensembl-genome.pl}
\end{itemize}

Those programs require to install a few Perl libraries as well as a
MySQL client on your machine. 

The first requirement is the \program{BioPerl} module, which has in
principle been installed in Chapter~\ref{chap:perl_modules}).

% To obtain BioPerl, you first need to install the program \program{git}
% (\url{http://git-scm.com/}). You can then execute the following
% instructions.

% \begin{lstlisting}
% ## For this example, we install Bioperl and EnsEMBL libraries 
% ## in $RSAT/lib, but you can install it in some other place
% mkdir -p $RSAT/lib
% cd $RSAT/lib


% ## Easy installation using GIT, protocol from http://www.bioperl.org/wiki/Using_Git
% git clone git://github.com/bioperl/bioperl-live.git

% ## To test the version of BioPerl that you are running, use:
% perl -MBio::Perl -le 'print Bio::Perl->VERSION;'
% \end{lstlisting}
% %%$

To obtain EnsEMBL \footnote{Full instructions at
  \url{http://useast.ensembl.org/info/docs/api/api_cvs.html}}.

\begin{lstlisting}

## Make sure you start from the right directory
cd $RSAT/lib

## Login on the EnsEMBL CVS server
cvs -d :pserver:cvsuser@cvs.sanger.ac.uk:/cvsroot/ensembl login
## (password is CVSUSER)

## Download the current version of EnsEMBL
cvs -d :pserver:cvsuser@cvs.sanger.ac.uk:/cvsroot/ensembl \
  checkout -r branch-ensembl-61 ensembl
## (you need to adapt the branch number, e.g. 61, as it updates).

## Download the current version of the EnsEMBL module for comparative genomics
cvs -d :pserver:cvsuser@cvs.sanger.ac.uk:/cvsroot/ensembl \
  checkout -r branch-ensembl-61 ensembl-compara
## (you need to adapt the branch number, e.g. 61, as it updates).
\end{lstlisting}
%%$ 

Adapt the 3 following lines in the RSAT configuration file
\file{RSAT\_config.props} to specify the actual path of the bioperl
and ensembl libraries on your computer. \emph{The path must be adapted
  to fit your local configuration}.

\begin{lstlisting}
ensembl=[RSAT_PARENT_PATH]/rsa-tools/lib/ensembl/modules
compara=[RSAT_PARENT_PATH]/rsa-tools/lib/ensembl-compara/modules
bioperl=[RSAT_PARENT_PATH]/rsa-tools/lib/bioperl-live
\end{lstlisting}

You also need to define the URL of the Ensembl database in that
configuration file:

\begin{lstlisting}
## EnsEMBL host
## Used by the EnsEMBL-accessing tools (retrieve-ensembl-seq,
## get-ensembl-genome).
## URL of the server for the EnsEMBL DB. By default, the
## main ensembl server is called, but a local server can be specified.
ensembl_host=ensembldb.ensembl.org
\end{lstlisting}

Finally, you need to include the BioPerl and Ensembl librairies in the
Perl module path (specified by the environment variable named
\$PERL5LIB).

If your shell is bash, add the following lines in the file
\file{.bashrc} at the root of your account.

\begin{lstlisting}
export PERL5LIB=${PERL5LIB}:${RSAT}/lib/ensembl/modules
export PERL5LIB=${PERL5LIB}:${RSAT}/lib/ensembl-compara/modules
export PERL5LIB=${PERL5LIB}:${RSAT}/lib/ensembl-variation/modules
\end{lstlisting}

If your shell is tcsh or chs, add the following lines in the file
\file{.cshrc} at the root of your account.

\begin{lstlisting}
setenv PERL5LIB ${PERL5LIB}:${RSAT}/lib/ensembl/modules
setenv PERL5LIB ${PERL5LIB}:${RSAT}/lib/ensembl-compara/modules
setenv PERL5LIB ${PERL5LIB}:${RSAT}/lib/ensembl-variation/modules
\end{lstlisting}



The EnsEMBL libraries also require the SQL client Perl module
\file{DBD::mysql}, as well as \file{DBI}, which can be installed with
\program{cpan} (for this you need root privileges). Note that the
installation of the CPAN module DBD:mysql requires a prior
installation of a MySQL client on your machine
(\url{http://dev.mysql.com/downloads/}).

To access EnsEMBL versions above 47, you need port 5306 to be opened
on your machine. This might require to consult the system
administrator of your network in order to ensure that the Firewall
accepts this port.

Detailed information about the EnsEMBL libraries can be obtained on
the EnsEMBL web site
(\footnote{\url{http://www.ensembl.org/info/using/api/api\_installation.html}}).



%%%%%%%%%%%%%%%%%%%%%%%%%%%%%%%%%%%%%%%%%%%%%%%%%%%%%%%%%%%%%%%%
% Downloading genomes from RSAT main server

%%%%%%%%%%%%%%%%%%%%%%%%%%%%%%%%%%%%%%%%%%%%%%%%%%%%%%%%%%%%%%%%
%%%% Downloading organisms from RSAT data repository
%%%%%%%%%%%%%%%%%%%%%%%%%%%%%%%%%%%%%%%%%%%%%%%%%%%%%%%%%%%%%%%%

\chapter{Downloading genomes}
\label{downloading_genomes}

\RSAT includes a series of tools to install and maintain the latest
version of genomes.

The most convenient way to add support for one or several organisms on
your machine is to use the programs \program{supported-organisms} and
\program{download-organism}.

Beware, the complete data required for a single genome may occupy
several hundreds of Mb, because \RSAT not only stores the genome
sequence, but also the oligonucleotide frequency tables used to
estimate background models, and the tables of BLAST hits used to get
orthologs for comparative genomics. If you want to install many
genomes on your computer, you should thus reserve a sufficient amount
of space.

\section{Original data sources}

Genomes supported on \RSAT were obtained from various sources.

Genomes can be installed either from the \RSAT web site, or from their
original sources.  

\begin{itemize}
\item NCBI/Genbank (\url{ftp://ftp.ncbi.nih.gov/genomes/})

\item ENSEMBL (\url{http://www.ensembl.org/})

\item The EBI genome directory (\url{ftp://ftp.ebi.ac.uk/pub/databases/genomes/Eukaryota/})

\end{itemize}

Other genomes can also be found on the web site of a diversity of
genome-sequencing centers.

\section{Requirement : wget}

The download of genomes relies on the application \program{wget},
which is part of linux distribution. \program{wget} is a ``web
aspirator'', which allows to download whole directories from ftp and
http sites. You can check if the program is installed on your machine.

\begin{lstlisting}
wget --help
\end{lstlisting}


This command should return the help pages for \program{wget}.  If you
obtain an error message (``command not found''), you need to ask your
system administrator to install it.

\section{Importing organisms from the \RSAT main server}

The simplest way to install organisms on our \RSAT site is to download
the RSAT-formatted files from the web server. For this, you can use a
web aspirator (for example the program \program{wget}). 

Beware, the full installation (including Mammals) requires a large
disk space (several dozens of Gb). You should better start installting
a small genome and test it before processing to the full
installation. We illustrate the approach with the genome of our
preferred model organism: the yeast \textit{Saccharomyces cerevisiae}.

\subsection{Obtaining the list of organisms supported on the \RSAT server}

By default, the program \program{supported-organisms} returns the list
of organisms supported on your local \RSAT installation. You can
however use the option \option{-server} to obtain the list of
organisms supported on a remote server.


\begin{lstlisting}
supported-organisms -server
\end{lstlisting}

The command can be refined by restricting the list to a given taxon of
interest.

\begin{lstlisting}
supported-organisms -server -taxon Fungi
\end{lstlisting}

You can also ask additional information, for example the date of the
last update and the source of each genome.

\begin{lstlisting}
supported-organisms -server -taxon Fungi -return last_update,source,ID
\end{lstlisting}


\subsection{Importing a single organism}


The command \command{download-organism} allows youj to download one or
several organisms. Beware, the complete data for a single genome may
occupy several tens of Megabytes (Bacterial genomes) or a few
Gigabases (Mammalian). Downloading tenomes thus requires a fast
Internet connection, and may take time. If possible, please download
genomes during the night (European time).

As a first step, we recommend to download the genome of the yeast
\org{Saccharomyces cerevisiae}, since this is the model organism used
in our tutorials.


\begin{lstlisting}
download-organism -v 1 -org Saccharomyces_cerevisiae
\end{lstlisting}

In principle, the download should start immediately, and last for a
few minutes.

After the task is completed, you can check if the configuration file
has been correctly updated by typing the command.

\begin{lstlisting}
supported-organisms
\end{lstlisting}

In principle, the following information should be displayed on your
terminal.

\result{Saccharomyces\_cerevisiae}

You can also add parameters to get specific information on the
supported organisms.

\begin{lstlisting}
supported-organisms -return ID,last_update
\end{lstlisting}


\subsection{Importing a few selected organisms}

The program \program{download-organism} can be launched with a list of
organisms by using iteratively the option \option{-org}.


\begin{lstlisting}
download-organism -v 1 -org Escherichia_coli_K12 -org Salmonella_typhi
\end{lstlisting}

\subsection{Importing all the organisms from a given taxon}

For comparative genomics, it is convenient to install on your server
all the organisms of a given taxon. For this, you can simply use the
option \option{-taxon} of \program{download-organism}.

Before doing this, it is wise to check the number of genomes that
belong to this taxon on the server.

\begin{lstlisting}
## Get the list of organisms belonging to the order "Enterobacteriales" on the server
supported-organisms -taxon Enterobacteriales -server

## Count the number of organisms
supported-organisms -taxon Enterobacteriales -server | wc -l
\end{lstlisting}

In Oct 2009, there are 94 Enterobacteriales supported on the \RSAT
server. Before starting the download, you should check two things:
\begin{enumerate}
\item Has your network a sufficient bandwidth to ensure the transfer
  in a reasonable time ?
\item Do you have enough free space on your hard drive to store all those genomes ? 
\end{enumerate}

If the answer to both questions is ``yes'', you can start the
download.

\begin{lstlisting}
download-organism -v 1 -taxon Enterobacteriales 
\end{lstlisting}


\section{Adding support for Ensembl genomes}

In addition to the genomes imported and maintained on your local \RSAT
server, the program \program{retrieve-ensembl-seq} allows you to
retrieve sequences for any organism supported in the Ensembl database
(\url{http://ensembl.org}).

For this, you first need to install the Bioperl and Ensembl Perl
libraries (see section \ref{sect:ensembl_libraries}).

\subsection{Handling genomes from Ensembl}

The first step to work with Ensembl genomes is to check the list of
organisms currently supported on their Web server.

\begin{lstlisting}
supported-organisms-ensembl
\end{lstlisting}

You can then get more precise information about a given organism
(build, chromosomes) with the command \program{ensembl-org-info}.

\begin{lstlisting}
ensembl-org-info -org Drosophila_melanogaster
\end{lstlisting}

Sequences can be retrieved from Ensembl with the command
\program{retrieve-ensembl-seq}. 

You can for example retrieve the 2kb sequence upstream of the
transcription start site of the gene \gene{PAX6} of the mouse. 


\begin{lstlisting}
retrieve-ensembl-seq.pl -org Mus_musculus -q PAX6 \
  -type upstream -feattype mrna -from -2000 -to -1 -nogene -rm \
  -alltranscripts -uniqseqs
\end{lstlisting}

Options

\begin{itemize}

\item \option{-type upstream} specifies that we want to collect the
  sequences located upstream of the gene (more procisely, upstream of
  the mRNA).

\item \option{-feattype mrna} indicates that the reference for computing
  coordinates is the mRNA. Since we collect upstream sequences, the
  5'most position of the mRNA has coordinate 0, and upstream sequences
  have negative coordinates. Note that many genes are annotated with
  multiple RNAs for different reasons (alternative splicing,
  alternative transcription start sites). By default, the program will
  return the sequences upstream of each mRNA annotated for the query
  gene.

\item \option{-nogene} clip the sequences to avoid overlapping the next
  upstream gene.

\item \option{-rm} repeat masking (important for pattern
  discovery). Repetitive sequences are replaced by \seq{N} characters.

\end{itemize} 





%%%%%%%%%%%%%%%%%%%%%%%%%%%%%%%%%%%%%%%%%%%%%%%%%%%%%%%%%%%%%%%%
% Tests for the command line tools
\chapter{Testing the command-line tools}

\section{Testing the access to the programs}

\subsection{Perl scripts}

From now on, you should be able to use the perl scripts from the
command line. To test this, run:

\begin{lstlisting}
random-seq -help
\end{lstlisting}


This should display the on-line help for the random sequence
generator.

\begin{lstlisting}
random-seq -l 200 -n 4
\end{lstlisting}

Should generate a random sequence of 200 nucleotides.


\subsection{Testing Perl graphical librairies}

\RSAT includes some graphical tools (\program{feature-map} and
\program{XYgraph}), which require a proper installation of Perl
modules.

\begin{description}
\item[GD.pm] Interface to Gd Graphics Library.
\item[PostScript::Simple]  Produce PostScript files from Perl.
\end{description}

To test if these modules are available on your machine, type.

\begin{lstlisting}
feature-map -help
\end{lstlisting}

If the modules are available, you should see the help message of the
program feature-map. If not, you will see an error message complaining
about the missing librairies. In such a case, ask your system
administrator to install the missing modules.

\subsection{Python scripts}

The \RSAT distribution includes some Python scripts. To test if they
are running correctly, you can try the proram \program{random-motif}.

\begin{lstlisting}
random-motif  -l 10 -c 0.85 -n 3
\end{lstlisting}

This command will generate 3 position-specific scoring matrix (PSSM)
of 10-columns with 85\% conservation of one residue in each column.

\subsection{C programs}

You can test the correct installation of the C programs with the
following command.

\begin{lstlisting}
random-seq -l 1000 -n 10 | count-words -l 2 -v 1 -2str -i /dev/stdin
\end{lstlisting}

The first program (\program{random-seq}) is a Perl script, which
generates a random sequence. The output is directly piped to the C
program \program{count-words}, which computes the frequencies and
occurrences of each dinucleotide.

\section{Testing genome installation}

We will now test if the genomes are correctly installed. You can
obtain the list of supported organisms with the command:

\begin{lstlisting}
supported-organisms
\end{lstlisting}


If this command returns no result, it means that genomes were either
not installed, or not correctly configured. In such a case, check the
directories in the \file{data/genomes} directory, and check that the
file \file{data/supported\_organisms.pl}.

Once you can obtain the list of installed organisms, try to retrieve
some upstream sequences. You can first read the list of options for the
\program{retrieve-seq} program.

\begin{lstlisting}
retrieve-seq -help
\end{lstlisting}


Select an organism (say \org{Saccharomyces cerevisiae}), and
retrieve all the start codons with the following options :

\begin{lstlisting}
retrieve-seq -org Saccharomyces_cerevisiae \
        -type upstream -from 0 -to +2 -all \
        -format wc -nocomment
\end{lstlisting}


This should return a set of 3 bp sequences, mostly ATG (in the case of
\org{Saccharomyces cerevisiae} at least)




%% \section{Further steps}

%% The installation is now finished, you can start the user's guide.

%% In case you would like to install additional genomes that are not
%% supported on \RSAT main server, the next chapter indicates yiou how to
%% proceed.


\chapter{Installing third-party programs}

\section{complementary programs for the analysis of regulatory
  sequences}

The \RSAT distribution only contains the programs developed by the
\RSAT \ team. 

A few additional programs, developed by third parties, can be
integrated in the package. All third-party programs may be loacated in
the directory \emph{bin} directory of the \RSAT distribution. In order
to obtain these programs, please download them from their original
site.

In particular, we recommend to install the following programs.

\begin{description}
\item[\program{vmatch}]: developed by Stefan Kurtz, is used used by the
program \program{purge-sequences}, for the detection of sequence repeats.

\item[\program{seqlogo}]:  developed by Thomas D. Schneider, is used
used by the program \program{convert-matrix} to generate logos. It can
be downloaded from $<$http://weblogo.berkeley.edu/$>$. \program{seqlogo} is
the command-line version of \program{WebLogo}.

Download the source code archive and uncompress it. Copy the following
 files to the directory \emph{bin} of your \RSAT distribution:
\program{seqlogo, logo.pm, template.pm} and \program{template.eps}.

\program{seqlogo} requires a recent version of \program{gs}
(ghostscript\urlref{http://www.ghostscript.com/}) to create PNG and
PDF output, and \program{ImageMagic's
  convert}\urlref{http://www.imagemagick.org/} to create GIFs.

\item[\program{patser}]: developed by Jerry Hertz, is used for
  matrix-based pattern matching.

\item[matrix-based pattern discovery]: several other pattern discovery
  programs have been embedded in the \RSAT program
  \program{multiple-family-analysis}:
\program{consensus} (Jerry Hertz),
\program{meme} (Tim Bailey),
\program{MotifSampler} (Gert Thijs),
\program{gibbs} (Andrew Neuwald).


\end{description}


\begin{table}
\begin{center}
\begin{tabular}{lll}
  \hline
  Program & author  & URL \\
  \hline
  vmatch & Stefan Kurtz & \footnotesize{\url{http://www.vmatch.de/}} \\
  seqlogo & Thomas Sneider & \footnotesize{\url{http://weblogo.berkeley.edu/}} \\
  patser & Jerry Hertz & \footnotesize{\url{ftp://ftp.genetics.wustl.edu/pub/stormo/Consensus/}} \\
  consensus & Jerry Hertz &  \footnotesize{\url{ftp://ftp.genetics.wustl.edu/pub/stormo/Consensus/}} \\
  meme & Tim Bailey & \footnotesize{\url{http://meme.sdsc.edu/}} \\
  MotifSampler & Gert Thijs & \footnotesize{\url{http://www.esat.kuleuven.ac.be/~thijs/download.html}} \\
%%  gibbs & Andrew Neuwald & \footnotesize{\url{ftp://ftp.ncbi.nih.gov/pub/neuwald/gibbs9\_95/}} \\
  \hline
\end{tabular}
\end{center}
\caption{\label{table:other_programs} Programs from other developers
  which are complementary to the \RSAT package.}
\end{table}

We particularly recommend the installation of \program{mkvtree} and
\program{vmatch} (Stefan Kurtz), because these programs are used by
the program purge-seq to discard redundant sequence fragments.

In order to add functionalities to \RSAT, install some or all of these
programs and include their binaries path rsa-tools/bin. If you are not
familiar with the installation of unix programs, ask assistance to
your system administrator.


\section{Recommended programs for the Network Analysis Tools (NeAT)}


For the Network Analysis Tools (NeAT), we recommend to install the
following programs, which offer complementary functionalities for the
analysis of networks/graphs.

Some programs come in the contrib directory of the \RSAT
distribution. Some others have to be downloaded from their original
distribution site.

To compile the program \program{floydwarshall} located in the
\file{contrib/floydwarshall} directory of \RSAT, use this command :

\begin{lstlisting}
gcc $RSAT/contrib/floydwarshall/floydwarshall.c -o $RSAT/bin/floydwarshall
\end{lstlisting}


In addition, the contributed programs REA and kWalks need to be installed.

To install kWalks, type:


\begin{lstlisting}
cd $RSAT/contrib/kwalks; 
tar xzvf kwalks.tgz; 
cd src; 
make clean; make
\end{lstlisting}

You can test taht the compilation worked by running the following
command.

\begin{lstlisting}
$RSAT/contrib/kwalks/bin/lkwalk
\end{lstlisting}
%% $

This should display the help message of \program{lkwalk}.

Check that the KWALKS\_ROOT variable in the RSAT config file
(\file{\$RSAT/RSAT\_config.props}) points to the correct path (it
should be the absolute path of \file{\$RSAT/contrib/kwalks/bin/}).

To install REA, type:

\begin{lstlisting}
cd $RSAT/contrib/REA; 
tar xzvf REA.tgz; 
rm -f *.o; 
make
\end{lstlisting}

You can test taht the compilation worked by running the following
command.

\begin{lstlisting}
$RSAT/contrib/REA/REA
\end{lstlisting}

This should display the help message of \program{REA}.

Check that the REA\_ROOT variable in the RSAT config file
(\file{\$RSAT/RSAT\_config.props}) points to the correct path (it
should be the absolute path of \file{\$RSAT/contrib/REA}).

%%%%%%%%%%%%%%%%%%%%%%%%%%%%%%%%%%%%%%%%%%%%%%%%%%%%%%%%%%%%%%%%
% Parsing and installing additional genomes

%%%%%%%%%%%%%%%%%%%%%%%%%%%%%%%%%%%%%%%%%%%%%%%%%%%%%%%%%%%%%%%%
%%%% Installing organisms
%%%%%%%%%%%%%%%%%%%%%%%%%%%%%%%%%%%%%%%%%%%%%%%%%%%%%%%%%%%%%%%%

\chapter{Installing organisms}


\RSAT includes a series of tools to install and maintain the latest
version of genomes.

\section{Original data sources}

Genomes supported on \RSAT were obtained from various sources.

Genomes can be installed either from the \RSAT web site, or from their
original sources.  

\begin{itemize}
\item NCBI/Genbank (\url{ftp://ftp.ncbi.nih.gov/genomes/})

\item ENSEMBL (\url{http://www.ensembl.org/})

\item The EBI genome directory (\url{ftp://ftp.ebi.ac.uk/pub/databases/genomes/Eukaryota/})

\end{itemize}

Other genomes can also be found on the web site of a diversity of
genome-sequencing centers.

\section{Requirement : wget}

The download of genomes relies on the application \program{wget},
which is part of linux distribution. \program{wget} is a ``web
aspirator'', which allows to downlaod whole directories from ftp and
http sites. You can check if the program is installed on your machine.

\begin{verbatim}
wget -help
\end{verbatim}

This command should return the help pages for \program{wget}.  If you
obtain an error message (``command not found''), you need to ask your
system administrator to install it.

\section{Importing organisms from the \RSAT main server}

The simplest way to install organisms on our \RSAT site is to download
the RSAT-formatted files from the web server. For this, you can use a
web aspirator (for example the program \program{wget}). 

Beware, the full installation (including Mammals) requires a large
disk space (several dozens of Gb). You should better start installting
a small genome and test it before processing to the full
installation. We illustrate the approach with one of the smallest
sequenced genome: \textit{Mycoplasma genitalium}.

To download the genome in your \RSAT folder, tye the following
command.

\begin{small}
\begin{verbatim}
cd $RSAT
wget -rNL http://rsat.scmbb.ulb.ac.be/rsat/data/genomes/Mycoplasma_genitalium/
\end{verbatim}
\end{small}

This will create a local mirror of the \RSAT data repository. You can
check the result by typing.

\begin{small}
\begin{verbatim}
ls -l $RSAT/rsat.scmbb.ulb.ac.be/rsat/data/genomes/Mycoplasma_genitalium/
\end{verbatim}
\end{small}

When the download is complete, move the newly transferred genome to
the data directory of your \RSAT installation.

\begin{small}
\begin{verbatim}
mv $RSAT/rsat.scmbb.ulb.ac.be/rsat/data/genomes/Mycoplasma_genitalium \
    $RSAT/data/genomes/
\end{verbatim}
\end{small}

You need now to declare the newly installed organism. 

\begin{small}
\begin{verbatim}
install-organism -v 1 -task config \
    -org Mycoplasma_genitalium -up_from -400 -up_to -1
\end{verbatim}
\end{small}

You can now check the configuration file.

\begin{small}
\begin{verbatim}
tail -20 $RSAT/data/supported_organisms.pl
\end{verbatim}
\end{small}

If the installation was successfull, you should see something like this : 

\begin{tiny}
\begin{verbatim}
#### Mycoplasma_genitalium      Mycoplasma genitalium   2006/01/04 22:08:42
$supported_organism{'Mycoplasma_genitalium'}->{'name'} = "Mycoplasma genitalium";
$supported_organism{'Mycoplasma_genitalium'}->{'data'} = "$RSA/data/genomes/Mycoplasma_genitalium";
$supported_organism{'Mycoplasma_genitalium'}->{'last_update'} = "2006/01/04 22:08:42";
$supported_organism{'Mycoplasma_genitalium'}->{'features'} = "$RSA/data/genomes/Mycoplasma_genitalium/genome/feature.tab";
$supported_organism{'Mycoplasma_genitalium'}->{'genome'} = "$RSA/data/genomes/Mycoplasma_genitalium/genome/contigs.txt";
$supported_organism{'Mycoplasma_genitalium'}->{'seq_format'} = "filelist";
$supported_organism{'Mycoplasma_genitalium'}->{'taxonomy'} = "Bacteria; Firmicutes; Mollicutes; Mycoplasmataceae; Mycoplasma";
$supported_organism{'Mycoplasma_genitalium'}->{'synonyms'} = "$RSA/data/genomes/Mycoplasma_genitalium/genome/feature_names.tab";
$supported_organism{'Mycoplasma_genitalium'}->{'up_from'} = -400;
$supported_organism{'Mycoplasma_genitalium'}->{'up_to'} = -1;

return 1;
\end{verbatim}
\end{tiny}


\section{Installing genomes from  their original source}

The parsing of genomes from their original data
sources is more tricky than the synchronization from the \RSAT server.

In principle, if you succeeded, with the protocol above, to obtain the
genomes from the \RSAT server, you don't need to proceed to new
installations yourself and you can skip the rest of this chapter.

\subsection{The \RSAT genome files}


\begin{enumerate}
\item genome sequence
\item feature table
\item list of names/synonyms
\end{enumerate}

\subsubsection{Genome sequence} 

The genome must be in raw format (text files containing the sequence
without any space or carriage return). If the organism contains
several chromosomes, there should be one separate file per contig
(chromosome). 

In addition, the genome directory must contain one file listing the
contig (chromosome) files. You can find an example in the directory
\file{\$RSAT/data/genomes/Saccharomyces\_cerevisiae/genome/}.


\subsubsection{Feature table}

A feature-table giving the basic information about genes. This is
a tab-delimited text file. Each row contains information about one
gene. The columns contain the following information: 
\begin{enumerate}

\item Identifier

\item Feature type (e.g. ORF, tRNA, ...)

\item Name

\item Chromosome. This must correspond to one of the sequence
identifiers from the fasta file.

\item Left limit

\item Right limit

\item Strand (D for direct, R for reverse complemet)

\item Description. A one-sentence description of the gene function.

\end{enumerate}

\subsubsection{Gene names (synonyms)}

Optionally, you can provide a synonym file, which contains two
columns:

\begin{enumerate}
\item ID. This must be one identifier found in the feature table
\item Synonym
\end{enumerate}

Multiple synonyms can be given for a gene, by adding several lines with
the same ID in the first column.

\subsubsection{Example}

\begin{verbatim}
cd $RSAT/data/genomes/Saccharomyces_cerevisiae/genome/

## The list of sequence files
cat contigs.txt

## The sequence files
ls -l *.raw

## The feature table
head -30 feature.tab

## The gene names/synonyms
head -30 feature_names.tab

\end{verbatim}


\section{Parsing genomes from NCBI/Genbank}

The easiest way to install an organism in \RSAT is to download the
complete genome files from the NCBI
\urlref{ftp://ftp.ncbi.nih.gov/genomes/}, and to parse it with the
program \program{parse-genbank.pl}.

\subsection{Downloading genomes from NCBI/Genbank}

\RSAT includes a makefile to download genomes from different sources.
We provide hereafter a protocol to create a download directory in your
account, and download genomes in this directory. Beware, genomes
require a lot of disk space, especially for those of higher
organisms. To avoid filling up your hard drive, we illustrate the protocol
with the smallest procaryote genome to date: \textit{Mycoplasma
  genitamlium}.


\begin{verbatim}
## Creating a directory for downloading genomes in your home account
cd $HOME
mkdir -p downloads
cd downloads

## Creating a link to the makefile which allows you to dowload genomes
ln -s $RSAT/makefiles/downloads.mk ./makefile
\end{verbatim}

We will now download a small genome from NCBI/Genbank. 

\begin{verbatim}
## Downloading one directory from NCBI Genbank
cd $HOME/downloads/
make one_genbank_dir GB_DIR=genomes/Bacteria/Mycoplasma_genitalium
\end{verbatim}


\subsection{Parsing genomes from NCBI/Genbank}

The program \program{parse-genbank.pl} extract genome information
(sequence, gene location, ...) from Genbank flat files, and exports
the result in a set of tab-delimited files.

\begin{verbatim}



\end{verbatim}


\section{Parsing genomes from EMBL files}

The program \program{parse-embl.pl} reads flat files in EMBL format,
and exports genome sequences and features (CDS, tRNA, ...) in
different files.

As an example, we can parse a yeast genome sequenced by the
``Genolevures'' project
\urlref{http://natchaug.labri.u-bordeaux.fr/Genolevures/download.php}.

Let us assume that you want to parse the genome of the species
\textit{Debaryomyces hansenii}.

Before parsing, you need to download the files in your account, 

\begin{itemize}
\item Create a directory for storing the EMBL files. The last level of
  the directory should be the name of the organism, where spaces are
  replaced by underscores. Let us assume that you store them in
  the directory \file{\$HOME/downloads/Debaryomyces\_hansenii}.

\item Download all the EMBL file for the selected organism. Save each
  name under its original name (the contig ID), followed by the
  extension \texttt{.embl})

\end{itemize}

We will check the content of this directory.

\begin{verbatim}
ls -1 $HOME/downloads/Debaryomyces_hansenii
\end{verbatim}

On my computer, it gives the following result

\begin{verbatim}
CR382133.embl
CR382134.embl
CR382135.embl
CR382136.embl
CR382137.embl
CR382138.embl
CR382139.embl
\end{verbatim}

The following instruction will parse this genome.

\begin{verbatim}
parse-embl.pl -v 1 -i  $HOME/downloads/Debaryomyces_hansenii
\end{verbatim}

If you do not specify the output directory, a directory is
automatically created by combining the current date and the organism
name.  The verbose messages will indicate you the path of this
directory, something like
\file{\$HOME/parsed\_data/embl/20050309/Debaryomyces\_hanseni}.


\subsection{Installing a genome in the main \RSAT directory}

Once the genome has been parsed, the simplest way to make it available
 for all the users is to install it in the \RSAT genome directory. You
 can already check the genomes installed in this directory.

\begin{verbatim}
ls -1 $RSAT/data/genomes/
\end{verbatim}

There is one subdirectory per organism. For example, the yeast data is
 in \file{\$RSAT/data/genomes/Saccharomyces\_cerevisiae/}. This
 directory is further subdivided in folders: \file{genome} and
 \file{oligo-frequencies}.

We will now create a directory to store data about
 Debaryomyces\_hansenii, and transfer the newly parsed genome in this
 directory.

\begin{verbatim}
## Create the directory
mkdir -p $RSAT/data/genomes/Debaryomyces_hansenii/genome

## Transfer the data in this directory
mv $HOME/parsed_data/embl/20050309/Debaryomyces_hanseni/* \
  $RSAT/data/genomes/Debaryomyces_hansenii/genome

## Check the transfer
ls -ltr $RSAT/data/genomes/Debaryomyces_hansenii/genome
\end{verbatim}

\subsection{Updating the configuration file}

The fact to add a directory is not sufficient for \RSAT to be aware of
the new organism. For this, we must update the configuration file. We
will also specify the default upstream sequence length. For a yeast
(\textit{Debaryomyces hansenii}), a good guess is 800bp (this is at
least the value I chose for \textit{Saccharomyces cerevisiae}).

\begin{verbatim}
install-organism -v 1 -org Debaryomyces_hansenii -task config -up_from -800

## Check the last lines of the configuration file
tail -15 $RSAT/data/supported_organisms.pl
\end{verbatim}

\section{Checking that the organism is installed properly}

To check the installation, start by checking whether your newly
installed now appears in the list of supported organisms.

\begin{verbatim}
supported-organisms
\end{verbatim}

Will give you a list of installed organisms.


\subsection{Retrieving sequences}

As soon as the configuration file has been updated, we should be in
state to retrieve sequences for the newly installed genome. We will
check this by retrieving a the start codons.

\begin{verbatim}
retrieve-seq -org Debaryomyces_hansenii -all -from 0 -to 2
\end{verbatim}

\subsection{Checking the composition of start codons}

Once the organism is found in your configuration, you need to check
whether sequences are retrieved properly. A good test for this is to
retrieve all the start codons, and check whether they are made of the
expected codons (mainly ATG, plus some alternative start codons like
GTG or TTG for bacteria).

We will now analyze the trinucleotide composition of the start
codons. In principle, al of them should be ATG for an eucaryote
organism.

\begin{verbatim}
retrieve-seq -org Debaryomyces_hansenii -all -from 0 -to 2 \
    | oligo-analysis -l 3 -1str -v 1 -return occ,freq -sort
\end{verbatim}

\subsection{Checking the start and stop codons with \program{install-organisms}}

The program \program{install-organisms} includes an option to
automatically check all the start and stop codons from a parsed
organism.

\begin{verbatim}
install-organism -v 1 -org Debaryomyces_hansenii -task start_stop
\end{verbatim}

You can then check the composition of the start and stop codons.

\begin{verbatim}
cd $RSAT/data/genomes/Debaryomyces_hansenii/genome/
more Debaryomyces_hansenii_start_codon_frequencies
more Debaryomyces_hansenii_stop_codon_frequencies
\end{verbatim}

The stop codons should be TAA, TAG or TGA, for any organism. For
eucaryotes, all start codons should be ATG. For some procaryotes,
alternative start codons (GTG, TGG) are frequent.

\subsection{Calibrating oligonucleotide and dyad frequencies with \program{install-organisms}}

The programs \program{oligo-analysis} and \program{dyad-analysis}
  require calibrated frequencies for the background models. These
  frequencies are calculated automatically with
  \program{install-organism}.

\textbf{Warning: } this task requires several hours of computation (a
few hours for small bacterial genomes, and several days for the human
genome).

\begin{verbatim}
install-organism -v 1 -org Debaryomyces_hansenii \
    -task allup,oligos,dyads,upstream_freq,clean
\end{verbatim}

\subsection{Installing a genome in your own account}

We describe below how this information should be formatted to be used
in rsa-tools.

In this chapter, we explain how to add support for an organism on your
local configuration of \RSAT. This assumes that you have the complete
sequence of a genome, and a table describing the predicted location of
genes.

First, prepare a directory where you will store the data for your
organism. For example:

\begin{verbatim}
mkdir -p $HOME/rsat-add/data/Mygenus_myspecies/
\end{verbatim}


One you have this information, start the program
\program{install-organism}. You will be asked to enter the information
needed for genome installation.

\section{Updating your local configuration}


\begin{itemize}
\item Modify the local config file

\item You need to define an environment variable called
  RSA\_LOCAL\_CONFIG, containing the full path of the local config
  file.

\end{itemize}


% Bibliography is not working yet. I have to debug
\bibliographystyle{plain}
\bibliography{rsat_bibliography}

\end{document}

%%%%%%%%%%%%%%%%%%%%%%%%%%%%%%%%%%%%%%%%%%%%%%%%%%%%%%%%%%%%%%%%

