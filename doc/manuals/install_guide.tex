%%%%%%%%%%%%%%%%%%%%%%%%%%%%%%%%%%%%%%%%%%%%%%%%%%%%%%%%%%%%%%%%
%
% Installation guide for regulatory Sequence Analysis Tools
%
%%%%%%%%%%%%%%%%%%%%%%%%%%%%%%%%%%%%%%%%%%%%%%%%%%%%%%%%%%%%%%%%

\documentclass{article}
%\documentstyle[makeidx]{book}
\makeindex
\usepackage{color}
\usepackage{times}
\usepackage{graphics}
\usepackage{latexsym}
\usepackage{makeidx}


%%%%%%%%%%%%%%%%%%%%%%%%%%%%%%%%%%%%%%%%%%%%%%%%%%%%%%%%%%%%%%%%
%%%%%%%%%%%%%%%%%%%%%%%%%%% commands %%%%%%%%%%%%%%%%%%%%%%%%%%%
\newcommand{\RSAT}{\textbf{\textit{RSAT}}}
\newcommand{\file}[1]{\textit{#1}}
\newcommand{\concept}[1]{\index{#1}\textsl{#1}}
\newcommand{\command}[1]{\begin{footnotesize}\begin{quote}\textcolor{blue}{\texttt{#1}}\end{quote}\end{footnotesize}}
\newcommand{\result}[1]{\begin{footnotesize}\begin{quote}\textcolor{green}{\texttt{#1}}\end{quote}\end{footnotesize}}
\newcommand{\program}[1]{\textbf{\textsl{#1}}}
\newcommand{\option}[1]{\texttt{#1}}
\newcommand{\email}[1]{\textit{#1}}

\newcommand{\address}[1]{\small{#1}}
\newcommand{\org}[1]{\textit{#1}}
\newcommand{\gene}[1]{\textit{#1}}

\newcommand{\url}[1]{\textit{#1}}
\newcommand{\urlref}[1]{\footnote{\textit{#1}}}

\newcommand{\scmbb}{
	Service de Conformation des Macromol\'{e}cules Biologiques et de Bioinformatique, \\
	Universit\'{e} Libre de Bruxelles, \\
	Campus Plaine, CP 263, Boulevard du Triomphe, B-1050 Bruxelles, Belgium. \\
	Tel: +32 2 650 2013 - Fax: +32 2 650 5425
}

%%%%%%%%%%%%%%%%%%%%%%%%%%%%%%%%%%%%%%%%%%%%%%%%%%%%%%%%%%%%%%%%
%%%%%%%%%%%%%%%%%%%%%%%%% environments %%%%%%%%%%%%%%%%%%%%%%%%%




\begin{document}

\title{Regulatory Sequence Analysis Tools \\
Installation guide}

\author{
	Jacques van Helden \\
	\email{jvanheld@ucmb.ulb.ac.be} \\
	\scmb 
}


%\address{\scmb}

\maketitle

%\newpage
%\tableofcontents
%\newpage

\section{Description}

This documents describes the installation procedure for the software
package \textbf{Regulatory Sequence Analysis Tools} (\RSAT).

\section{Requirements}

\subsection{Operating system}

\RSAT is a unix-based package. It has been installed successfully on
the following unix systems.

\begin{enumerate}
\item Linux
\item Mac OSX
\item Sun Solaris
\item Dec Alpha
\item cygwin (under MS Windows 98)
\end{enumerate}

\RSAT is not compatible with any version of Microsoft Windows and we
have no intention to make it compatible in a foreseeable future. Since
most programs are written in perl, part of them might run under
windows, but some others will certainly not, because they include
calls to unix system commands.

\subsection{perl}

Almost all the programs in \RSAT are written in perl version 5.0 or
later.

\section{Installation}

\RSAT can be distributed either as a compressed archive, or via the CVS
server. The CVS distribution greatly facilitates updates.

\subsection{Installation from the CVS repository}

Before being able to retrieve \RSAT from the CVS repository, you need
an account on our server. For this, please contact Jacques van Helden
(\email{jvanheld@ucmb.ulb.ac.be}).

\subsubsection{First installation}

The following command should be used the first time you retrieve the
tools from the server:
\begin{verbatim}
cvs -d mylogin@rubens.ulb.ac.be:/rubens/dsk2/cvs co rsa-tools
\end{verbatim}

This will create a directory \file{rsa-tools} on your computer, and
store the programs in it. Note that at this stage the programs are not
yet functional, because you still need to install genomes, which are
not included in the CVS distribution.

\subsubsection{Updates}

Once the tools have been retrieved, you can obtain updates very
easily. For this, you need to change your directory to the rsa-tools
directory, and use the \texttt{cvs} command in the following way.

\begin{verbatim}
cd rsa-tools
cvs update .
\end{verbatim}

\subsection{Installation from a compresed archive}

\begin{enumerate}

\item Uncompress the archive containing the programs. 
\begin{verbatim}
tar -xpzf rsat_yyyymmdd.tgz
\end{verbatim}

where yyyymmdd stands for the version number (delivery date).

\end{enumerate}

\section{Adding \RSAT to your path}

\begin{enumerate}

\item Create an environment variable named RSAT and containing the
path of rsa-tools. For example, assuming \RSAT have been installed in
the directory \texttt{/home/myaccount/rsa-tools}, and your shell is
bash:

\begin{verbatim}
export RSAT=/home/myaccount/rsa-tools
\end{verbatim}

\item add the path of rsa-tools/perl-scripts and binaries to your path.

\begin{verbatim}
export PATH=${PATH}:${RSAT}/bin
export PATH=${PATH}:${RSAT}/perl-scripts
\end{verbatim}

\end{enumerate}


\section{Initializing the directories}

In addition to the programs, the installation of rsa-tools requires
the creation of a few directories for storing data, access logs (forr
the web server), and temporary files.  

The distribution includes a series of make scripts which will
facilitate this step. You just need go to the rsa-tools directory, and
start the appropriate make file.

\begin{verbatim}
cd rsa-tools
make -f makefiles/init_RSAT.mk init
\end{verbatim}



\section{Installing genomes}

Genomes are distributed via the HTTP server. 

\url{http://rsat.ulb.ac.be/rsat/data/genomes/}

Download the genomes you need and store them in the directory

\file{rsa-tools/public\_html/data/genomes}

You also need the file which provides the list of supported genomes.

\url{http://rsat.ulb.ac.be/rsat/data/supported\_organisms.pl}

This file must be stored in your directory

\file{rsa-tools/public\_html/data/}


\section{Configuring \RSAT}

The \RSAT distribution comes with a template configuration file.

\file{rsa-tools/RSA.config}

Open this file with a text editor, and edit the first variables
according to your local configuration.

\section{Testing the installation}

\subsection{Testing the access to perl scripts}

From now on, you should be able to use the perl scripts from the
command line. To test this, run: 

\begin{verbatim}
random-seq -help
\end{verbatim}

This should display the on-line help for the random sequence
generator. 

\begin{verbatim}
random-seq -l 200 -r 4 -a a:t 0.3 c:g 0.2
\end{verbatim}

Should generate a random sequence.

\subsection{Testing genome installation}

We will now testif the genomes are correctly installed. You will
obtain the list of supported organisms with the command:

\begin{verbatim}
retrieve-seq -help
\end{verbatim}

Select an organism (say \organism{Saccharomyces cerevisiae}), and
retrieve all the start codons with the following options :

\begin{verbatim}
retrieve-seq -org Saccharomyces_cerevisiae \
        -type upstream -from 0 -to +2 -all \
        -format wc -nocomment 
\end{verbatim}

This should return a set of 3 bp sequences, mostly ATG (in the case of
\organism{Saccharomyces cerevisiae} at least)

\subsection{Testing the graphical scripts}

\RSAT includes two graphical tools, \program{feature-map} and
\program{XYgraph}. These tools require the perl library GD.pm. To test
if this library has been installed on your machine, type.

\begin{verbatim}
feature-map -help
\end{verbatim}

If you receive the correct help message, it means that the library is
intalled.


\section{Further steps}

The installation is now finished, you can start the user's guide. 

\end{document}

%%%%%%%%%%%%%%%%%%%%%%%%%%%%%%%%%%%%%%%%%%%%%%%%%%%%%%%%%%%%%%%%

