
%%%%%%%%%%%%%%%%%%%%%%%%%%%%%%%%%%%%%%%%%%%%%%%%%%%%%%%%%%%%%%%%
%%%% Introduction
%%%%%%%%%%%%%%%%%%%%%%%%%%%%%%%%%%%%%%%%%%%%%%%%%%%%%%%%%%%%%%%%
\section{Introduction}

This tutorial aims at introducing how to use Regulatory Sequence
Analysis Tools (\RSAT) directly from the unix shell.

\RSAT is a package combining a series of specialized programs for the
detection of regulatory signals in non-coding sequences. A variety of
tasks can be performed: retrieval of upstream or downstream sequences,
pattern discovery, pattern matching, graphical representation of
regulatory regions, sequence conversions, \ldots.

A web interface has been developed for the most common tools, and is
freely available for academic users.

\url{http://www.ucmb.ulb.ac.be/bioinformatics/rsa-tools/}

All programs can also be used directly from the unix shell. The shell
access is less intuitive than the web interface, but is very
convenient for automatizing repetitive tasks.

This tutorial was written by Jacques van Helden
(\email{jvanheld@ucmb.ulb.ac.be}).  Unless otherwise specified, the
programs presented here were written by Jacques van Helden.
