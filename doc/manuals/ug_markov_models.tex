%%%%%%%%%%%%%%%%%%%%%%%%%%%%%%%%%%%%%%%%%%%%%%%%%%%%%%%%%%%%%%%%
%% MARKOV MODELS
%%%%%%%%%%%%%%%%%%%%%%%%%%%%%%%%%%%%%%%%%%%%%%%%%%%%%%%%%%%%%%%%

\section{Introduction to Markov order}

Markov chain models involve local dependencies between successive
residues.

A Markov model of order $m$ means that the
probability of the residue at each position $i$ of the sequence
depends on the $m$ preceding residues.

\subsection{Transition frequency tables}

Markov models are described by transition frequencies $P(r|W_m)$,
i.e. the probability to osberve residue $r$ at a certain position,
depending on the preceding word $W_m$ of size $m$.

\subsection{Oligonucleotide frequency tables}

\RSAT allows to derive organism-specific Markov models from
oligonucleotide frequency tables.

Pre-calibrated oligonucleotide frequency tables are stored in the form
of oligonucleotide frequency tables (see chapter on pattern
discovery).

The calibration tables for \org{Escherichia
  coli K12} can be found in the following directory.

\begin{small}
\begin{verbatim}
ls -ltr $RSAT/data/genomes/Escherichia\_coli\_K12/oligo-frequencies
\end{verbatim}
\end{small}

For example, the file
\file{4nt\_upstream-noorf\_Escherichia\_coli\_K12-1str.freq.gz}
indicates the tetranucleotide frequencies for all the upstream
sequences of \org{E.coli}.

\begin{small}
\begin{verbatim}
cd $RSAT/data/genomes/Escherichia_coli_K12/oligo-frequencies/

## Have a look at the content of the 4nt frequency file
gunzip -c 4nt_upstream-noorf_Escherichia_coli_K12-1str.freq.gz | more
\end{verbatim}
\end{small}

Transition frequencies are automatically derived from the table of
oligonucleotide frequencies, but one should take care of the fact
that, order to generate a random sequence with a Markov model of order
$m$, we need to use the frequency tables for oligonucleotides of size
$m+1$.

We can illustrate this convertion by converting the table of
dinucleotide frequencies into a transition matrix of first order. each
row of the matrix indicates the prefix $W_m$, and each column the
following residue $r$. For a Markov model of order 1, the prefixes are
single residues.

\begin{small}
\begin{verbatim}
convert-background-model \
  -i 2nt_upstream-noorf_Escherichia_coli_K12-1str.freq.gz  \
  -from oligo-analysis -to tab
\end{verbatim}
\end{small}

We can now obtain a Markov model of 2nd order, from the table of
trinucleotide frequencies.

\begin{small}
\begin{verbatim}
convert-background-model \
  -i 3nt_upstream-noorf_Escherichia_coli_K12-1str.freq.gz  \
  -from oligo-analysis -to tab
\end{verbatim}
\end{small}

The same operation can be extended to higher order markov models.
